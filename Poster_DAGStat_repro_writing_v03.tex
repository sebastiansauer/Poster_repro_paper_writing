\documentclass[final,hyperref={pdfpagelabels=false}]{beamer}
\mode<presentation>
  {
  %  \usetheme{Berlin}
  \usetheme{Dreuw}
  }
  \usepackage{times}
  \usepackage{amsmath,amsthm, amssymb, latexsym, tikz}

\usepackage{multicol}
\usetikzlibrary{arrows,shapes}
  
  \boldmath
  \usepackage[english]{babel}
  \usepackage[latin1]{inputenc}
  \usepackage[orientation=portrait,size=a0,scale=1.4,debug]{beamerposter}

  %%%%%%%%%%%%%%%%%%%%%%%%%%%%%%%%%%%%%%%%%%%%%%%%%%%%%%%%%%%%%%%%%%%%%%%%%%%%%%%%%5
  \graphicspath{{figures/}}
  \title{Making research reproducible \\using literate programming}
  \author{Sebastian Sauer, Sandra S\"ulzenbr\"uck, Yvonne Ferreira, and Christoph Kurz}
  \institute{FOM University of Applied Sciences, Helmholtz Zentrum M\"unchen}

  \date{March 2014}


  %%%%%%%%%%%%%%%%%%%%%%%%%%%%%%%%%%%%%%%%%%%%%%%%%%%%%%%%%%%%%%%%%%%%%%%%%%%%%%%%%5
  \begin{document}
  \begin{frame}{} 
    
      \begin{columns}[t]
      
      \begin{column}{.48\linewidth}
      
        \begin{block}{Reproducibility: What and why?}
      There is an increasing concern about the reliability of research results \cite{Peng2015}.
    Precisely, it has been found that many published results cannot be replicated \cite{OpenScienceCollaboration2015}.
    In parts, this can be due to the fact that it is often hardly possible for independent researchers to
    confirm the results of a research paper. Thus, making research more reproducible seems a pressing need  \cite{Peng2015}.
    Here, we present some ``recipe'' for making (your) research paper more reproducible using literate programming.
    Literate programming refers to weaving programming code (ie., statistical calculations) with the paper text.        
    
    \end{block}
      
      
      
            \begin{block}{Workflow for reproducible paper writing}
         
      \begin{minipage}[t]{0.45\textwidth}
      %Workflow, in general
      %\vspace{2cm}
      \tikzstyle{format} = [rectangle, draw, text centered, rounded corners]
      \tikzstyle{medium} = [rectangle, draw, thin]
      \tikzstyle{line} = [draw, -latex']
      \begin{tikzpicture}[node distance=7cm, auto]
          % We need to set a bounding box first. Otherwise the diagram
          % will change position for each frame.
          %\path[use as bounding box] (-5,0) rectangle (10,10);
        
        \node [format] (idea) {idea};
        \node [format, below of = idea] (notes) {notes};
        \node [format, below of = notes, align = center] (manuscript) {unformated\\manuscript};
        \node [format, below of = manuscript,  align = center] (paper) {typesetted\\paper};
        \node [medium, left of = manuscript, xshift = -3cm] (R)  {calculations};
        \node [medium, above left  of = manuscript, yshift = 1cm, align = center] (vc) {version\\control};
        \node [medium, below left of = manuscript,  xshift = -3cm, yshift = -3cm]  (collab) {collaborators};
        \path [line] (idea) -- node [align=center] {\emph{think,}\\\emph{scribble}} (notes) ;
        \path [line] (notes) --  node [align=center] {\emph{hack}\\\emph{text}} (manuscript);
        \path [line] (manuscript) -- node {\emph{format}}(paper);     
        \path [line, below=7cm] (R) --  node [align=center] {\emph{calc}\\\emph{data}}(manuscript);  
        \path [line, left] (vc) -- node {\emph{track changes}} (manuscript);
        \path [line] (collab) -- node [align=center, near start] {\emph{give}\\\emph{input}} (manuscript);
      \end{tikzpicture}
     \end{minipage}	
     %
     % now with software tools
     \begin{minipage}[t]{0.50\textwidth}
     \begin{tikzpicture}[node distance=7cm, auto]
          % We need to set a bounding box first. Otherwise the diagram
          % will change position for each frame.
          %\path[use as bounding box] (-5,0) rectangle (10,10);
          \tikzstyle{format} = [rectangle, draw, text centered, rounded corners]
      \tikzstyle{medium} = [rectangle, draw, thin]
      \tikzstyle{line} = [draw, -latex']
        \node [format] (idea) {idea};
        \node [format, below of = idea,  align = center] (notes) {Piece\\of Paper};
        \node [format, below of = notes, align = center] (manuscript) {ugly\\.txt file};
        \node [format, below of = manuscript,  align = center] (paper) {polished\\.pdf file};
        \node [medium, left of = manuscript, xshift = -3cm] (R)  {R-code};
        \node [medium, above left  of = manuscript, yshift = 1cm, align = left] (vc) {git};
        \node [medium, below left of = manuscript,  xshift = -3cm, yshift = -3cm]  (collab) {collaborators};
        \path [line] (idea) -- node [align=center]{\emph{think,}\\\emph{scribble} }(notes) ;
        \path [line] (notes) --  node [align=center]{\emph{hack}\\\emph{.md file}}  (manuscript);
        \path [line] (manuscript) -- node [align=center]{\emph{apply}\\\emph{.tex style}}(paper);     
       \path [line, below=7cm] (R) --  node [align=center] {\emph{weave}\\\emph{in}}(manuscript);  
        \path [line, left] (vc) -- node {\emph{track changes}} (manuscript);
        \path [line] (collab) -- node [align=center, near start] {\emph{give}\\\emph{input}} (manuscript);
      \end{tikzpicture}
     \end{minipage}	
     \\
    The \emph{diagram} shows the workflow \emph{in theory}. The \emph{right} diagram exemplifies useful tools for each step.
     
    \end{block}


   
     \begin{block}{What makes the researcher happy}
    
\begin{tabular}{rllll}
  \hline
 & Word & Latex & Markdown & WebApps \\ 
  \hline
easy &  $\bullet\bullet\bullet$ & $\bullet$ & $\bullet\bullet$ & $\bullet\bullet$ \\ 
  beautiful & $\bullet$ & $\bullet\bullet\bullet$ & $\bullet\bullet\bullet$ &  $\bullet\bullet$ \\ 
  literal & $\bullet$ & $\bullet\bullet\bullet$ & $\bullet\bullet\bullet$ & $\bullet$ \\ 
  readable & $\bullet$ &  $\bullet\bullet$ & $\bullet\bullet\bullet$ & $\bullet\bullet\bullet$ \\ 
  versionized & $\bullet$ & $\bullet\bullet\bullet$ & $\bullet\bullet\bullet$ & $\bullet\bullet\bullet$ \\ 
  citable & $\bullet\bullet\bullet$ & $\bullet\bullet\bullet$ & $\bullet\bullet\bullet$ & $\bullet\bullet\bullet$ \\ 
  flexible & $\bullet\bullet\bullet$ &  $\bullet\bullet$ & $\bullet$ &  $\bullet\bullet$ \\ 
   \hline
   $\Sigma$ & 13 & 17 & 18 & 16\\
   \hline
\end{tabular}
\bigskip

This table compares some (subjective) criteria of what a researcher needs for writing a paper in a reproducible and efficient way. Note: \emph{Word} refers not only to MS Word, but to similar WYSIWYG text processors as well. \emph{WebApps} refer to scholarly writing tools such as \emph{Authorea}. \emph{Markdown} refers to the \emph{pandoc} dialect and extensions of markdown.
   
        \end{block}



  \begin{block}{Software tools for reproducible writing}
    
\begin{tabular}{rllll}
  \hline
 &  & Name & Description  \\ 
  \hline



   \hline
\end{tabular}
\bigskip

jljkl\"o   
        \end{block}



  
      
      \end{column}
      
      
      
      
      \begin{column}{.48\linewidth}
        \begin{block}{Introduction}
          \begin{itemize}
          \item some items and $\alpha=\gamma, \sum_{i}$
          \item some items
          \item some items
          \item some items
          \end{itemize}
          $$\alpha=\gamma, \sum_{i}$$
        \end{block}

        \begin{block}{Introduction}
          \begin{itemize}
          \item some items
          \item some items
          \item some items
          \item some items
          \end{itemize}
        \end{block}

        \begin{block}{Introduction}
          \begin{itemize}
          \item some items and $\alpha=\gamma, \sum_{i}$
          \item some items
          \item some items
          \item some items
          \end{itemize}
          $$\alpha=\gamma, \sum_{i}$$
        \end{block}
      \end{column}
    \end{columns}
  \end{frame}
\end{document}


%%%%%%%%%%%%%%%%%%%%%%%%%%%%%%%%%%%%%%%%%%%%%%%%%%%%%%%%%%%%%%%%%%%%%%%%%%%%%%%%%%%%%%%%%%%%%%%%%%%%
%%% Local Variables: 
%%% mode: latex
%%% TeX-PDF-mode: t
%%% End:
